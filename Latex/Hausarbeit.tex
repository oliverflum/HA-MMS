\documentclass[12pt,a4paper,bibliography=totocnumbered,listof=totocnumbered]{scrartcl}
\usepackage[ngerman]{babel}
\usepackage[utf8]{inputenc}
\usepackage{amsmath}
\usepackage{amsfonts}
\usepackage{amssymb}
\usepackage{graphicx}
\usepackage{fancyhdr}
\usepackage{tabularx}
\usepackage{geometry}
\usepackage{setspace}
\usepackage[right]{eurosym}
\usepackage[printonlyused]{acronym}
\usepackage{subfig}
\usepackage{floatflt}
\usepackage[usenames,dvipsnames]{color}
\usepackage{colortbl}
\usepackage{paralist}
\usepackage{array}
\usepackage{titlesec}
\usepackage{parskip}
\usepackage[right]{eurosym}
\usepackage{picinpar}
\usepackage{cite}
\usepackage[subfigure,titles]{tocloft}
\usepackage[pdfpagelabels=true]{hyperref}

\renewcommand{\familydefault}{\sfdefault}
\usepackage{listings}
\lstset{basicstyle=\footnotesize, captionpos=b, breaklines=true, showstringspaces=false, tabsize=2, frame=lines, numbers=left, numberstyle=\tiny, xleftmargin=2em, framexleftmargin=2em}
\makeatletter
\def\l@lstlisting#1#2{\@dottedtocline{1}{0em}{1em}{\hspace{1,5em} Lst. #1}{#2}}
\makeatother

\geometry{a4paper, top=27mm, left=30mm, right=20mm, bottom=35mm, headsep=10mm, footskip=12mm}

\hypersetup{unicode=false, pdftoolbar=true, pdfmenubar=true, pdffitwindow=false, pdfstartview={FitH},
	pdftitle={Hausarbeit},
	pdfauthor={Oliver Flum (377780), Florian Kleeblatt (375911)	},
	pdfsubject={Hausarbeit},
	pdfcreator={\LaTeX\ with package \flqq hyperref\frqq},
	pdfproducer={pdfLaTeX \the\pdftexversion.\pdftexrevision},
	pdfkeywords={Hausarbeit},
	pdfnewwindow=true,
	colorlinks=true,linkcolor=black,citecolor=black,filecolor=magenta,urlcolor=black}
%\pdfinfo{/CreationDate (D:20110620133321)}
%% Option 'defaultfam'
%% only if the base font of the document is to be sans serif



\begin{document}

\titlespacing{\section}{0pt}{12pt plus 4pt minus 2pt}{-6pt plus 2pt minus 2pt}

% Kopf- und Fusszeile
\renewcommand{\sectionmark}[1]{\markright{#1}}
\renewcommand{\leftmark}{\rightmark}
\pagestyle{fancy}
\lhead{}
\chead{}
\rhead{\thesection\space\contentsname}
\lfoot{Virtual Augmented Reality}
\cfoot{}
\rfoot{\ \linebreak Seite \thepage}
\renewcommand{\headrulewidth}{0.4pt}
\renewcommand{\footrulewidth}{0.4pt}

% Vorspann
\renewcommand{\thesection}{\Roman{section}}
\renewcommand{\theHsection}{\Roman{section}}
\pagenumbering{Roman}

% ----------------------------------------------------------------------------------------------------------
% Titelseite
% ----------------------------------------------------------------------------------------------------------
\thispagestyle{empty}
\begin{center}
	\vspace*{2cm}
	\Large
	\textbf{Fakultät V}\\
	\textbf{Institut für Psychologie und Arbeitswissenschaft}\\
	\vspace*{2cm}
	\Huge
	\textbf{Hausarbeit}\\
	\vspace*{0.5cm}
	\large
	Mensch-Maschine-Systeme I\\
	\vspace*{1cm}
	\textbf{Virtual Augmented Reality}\\
	\vspace*{2cm}

	\vfill
	\normalsize
	\newcolumntype{x}[1]{>{\raggedleft\arraybackslash\hspace{0pt}}p{#1}}
	\begin{tabular}{x{6cm}p{7.5cm}}
		\rule{0mm}{5ex}\textbf{Autor:} & Florian Kleeblatt (375911)\newline Oliver Flum (377780) \\
		\rule{0mm}{5ex}\textbf{Prüfer:} & Prof. Dr.-Ing. Matthias Rötting\\
		\rule{0mm}{5ex}\textbf{Abgabedatum:} & 19.03.2017\\
	\end{tabular}
\end{center}
\pagebreak

% ----------------------------------------------------------------------------------------------------------
% Abstract
% ----------------------------------------------------------------------------------------------------------
%\setcounter{page}{1}
%\onehalfspacing
%\titlespacing{\section}{0pt}{12pt plus 4pt minus 2pt}{2pt plus 2pt minus 2pt}
%\rhead{ABSTRACT}
%\section{Abstract}
%Nach mehreren unausgereiften Versuchen seit den frühen 90er Jahren hat die Virtual Reality in den letzten Jahren eine %Renaissance erlebt. Ebenso liegt auch Augmented Reality im Aufwind.
%Diese neuen Technologien eröffnen neue Möglichkeiten der Interaktion mit digitalen Medien, die die klassische Desktop-%Metapher transzendieren.
%Da die momentan verfügbaren Systeme sich hauptsächlich auf Unterhaltungselektronik fokussieren, soll diese Arbeit sich mit %potentiellen Anwendungen in einem professionellen Umfeld beschäftigen.
%Obwohl diese Anwendungen vielseitige Betrachtungen wie ablenkungsfreies Arbeiten oder intuitive GUI-Gestaltung erlauben, %soll diese Arbeit sich auf die Bewertung von einer zwischen physikalischer Welt und virtuellen Objekten kompatiblen %Gestaltung beschränken.
%Ziel ,des in der Arbeit beschriebenen Versuchs, ist die Beantwortung der Frage, ob eine kompatible Gestaltung von Virtual %und Augmented Reality Umgebungen eine produktivere Arbeit mit digitalen Dokumenten erlaubt.

\pagebreak

% ----------------------------------------------------------------------------------------------------------
% Verzeichnisse
% ----------------------------------------------------------------------------------------------------------
% TODO Typ vor Nummer
\renewcommand{\cfttabpresnum}{Tab. }
\renewcommand{\cftfigpresnum}{Abb. }
\settowidth{\cfttabnumwidth}{Abb. 10\quad}
\settowidth{\cftfignumwidth}{Abb. 10\quad}

\titlespacing{\section}{0pt}{12pt plus 4pt minus 2pt}{2pt plus 2pt minus 2pt}
\singlespacing
\rhead{INHALTSVERZEICHNIS}
\renewcommand{\contentsname}{I Inhaltsverzeichnis}
\phantomsection
\addcontentsline{toc}{section}{\texorpdfstring{I \hspace{0.35em}Inhaltsverzeichnis}{Inhaltsverzeichnis}}
\addtocounter{section}{1}
\tableofcontents
\pagebreak
\rhead{VERZEICHNISSE}
\listoffigures
\pagebreak
\listoftables
%\pagebreak
\renewcommand{\lstlistlistingname}{Listing-Verzeichnis}
{\labelsep2cm\lstlistoflistings}
\pagebreak

% ----------------------------------------------------------------------------------------------------------
% Abkürzungen
% ----------------------------------------------------------------------------------------------------------
\section{Abkürzungsverzeichnis}
\begin{acronym}[Bash]
 \acro{VR}{Virtual Reality}
 \acro{AR}{Augmented Reality}
 \acro{VAR}{Virtual Augmented Reality}
 \acro{HUD}{Head-Up-Display}	

\end{acronym}
\newpage
% ----------------------------------------------------------------------------------------------------------
% Inhalt
% ----------------------------------------------------------------------------------------------------------
% Abstände Überschrift
\titlespacing{\section}{0pt}{12pt plus 4pt minus 2pt}{-6pt plus 2pt minus 2pt}
\titlespacing{\subsection}{0pt}{12pt plus 4pt minus 2pt}{-6pt plus 2pt minus 2pt}
\titlespacing{\subsubsection}{0pt}{12pt plus 4pt minus 2pt}{-6pt plus 2pt minus 2pt}

% Kopfzeile
\renewcommand{\sectionmark}[1]{\markright{#1}}
\renewcommand{\subsectionmark}[1]{}
\renewcommand{\subsubsectionmark}[1]{}
\lhead{Kapitel \thesection}
\rhead{\rightmark}

\onehalfspacing
\renewcommand{\thesection}{\arabic{section}}
\renewcommand{\theHsection}{\arabic{section}}
\setcounter{section}{0}
\pagenumbering{arabic}
\setcounter{page}{1}
% ----------------------------------------------------------------------------------------------------------
% Eimnleitung
% ----------------------------------------------------------------------------------------------------------
\section{Einleitung}
Nach mehreren unausgereiften Versuchen seit den frühen 90er Jahren hat die Virtual Reality in den letzten Jahren eine Renaissance erlebt. Ebenso liegt auch Augmented Reality\ac{AR}im Aufwind. Diese neuen Technologien eröffnen neue Möglichkeiten der Interaktion mit digitalen Medien, die die klassische Desktop-Metapher transzendieren. Da die momentan verfügbaren Systeme sich hauptsächlich auf Unterhaltungselektronik fokussieren, soll diese Arbeit sich mit potentiellen Anwendungen in einem professionellen Umfeld beschäftigen. Obwohl diese Anwendungen vielseitige Betrachtungen wie ablenkungsfreies Arbeiten oder intuitive GUI-Gestaltung erlauben, soll diese Arbeit sich auf die Bewertung von einer zwischen physikalischer Welt und virtuellen Objekten kompatiblen Gestaltung beschränken. Ziel, des in der Arbeit beschriebenen Versuchs, ist die Beantwortung der Frage, ob eine kompatible Gestaltung von Virtual und Augmented Reality Umgebungen eine produktivere Arbeit mit digitalen Dokumenten erlaubt.
% ----------------------------------------------------------------------------------------------------------
% Virtual Reality
% ----------------------------------------------------------------------------------------------------------
\section{Technik}
Dieses Kapitel gibt einen Überblick über die Konzepte der Virtual- und Augmented-Reality, sowie ein Vorstellung wie deren Vermischung aussehen kann. Neben den abstrakten Konzepten werden außerdem einige konkrete Implementierungen vorgestellt. Um eine Unterscheidung zu ermöglichen, wird auf das Kontinuum von Milgram und Kishino aufgebaut. Es wird dabei der Grad der Realität unterschieden (vgl Abbildung %TODO Referenzz zu Abbildung einführen
%TODO Abbildung einfügen
). Zum einen Ende gibt es die vollkommene Realität und das Gegenteil am anderen Ende ist die vollkommene Virtualität. Im Bereich dazwischen befindet sich die Mixed Reality. Die Virtual Reality liegt im Bereich der Virtualität und die Augmented Realität ist näher an den Bereich der Realität angelehnt. Vermischte Bereiche wie Augmented-Virtuality befinden sich im Bereich der Mixed-Reality.
\subsection{Virtual Reality}
\subsubsection{Allgemein}
Unter Virtual Reality (\ac{VR}) wird die Simulation von Sinneseindrücken mit dem Ziel eine künstliche Welt für den Menschen erlebbar zu machen verstanden. Im Allgemeinen steht dabei vor allem die visuelle Wahrnehmung im Vordergrund. Sekundär werden auch akustische, haptische bzw. taktile und akustische berücksichtigt. Auch Geruchs- und Geschmackssimulation sind Gegenstand der Entwicklung, aber finden eher wenig tatsächliche Anwendung.
Der Begriff ‚Virtual Reality‘ wurde durch den 1982 erschienen Science-Fiction-Roman ‚The Judas Mandala‘. Ein Gebiet der Forschung war die Virtual Reality bereits wesentlich früher.
Das Sensorama erlaubte eine nicht interaktive virtuelle Realität bei der den Benutzern die 
Betrachtung eines Filmes in stereoskopischer Optik, mit binereuralem Ton und künstlichem Wind und Geruch. Ein konkret ausformulierte Beschreibung von Virtual Reality findet sich erstmals in ‚The Ultimate Display‘ von Ivan Sutherland im Jahr 1965.

\begin{minipage}{\linewidth}
\vspace{1em}
	\centering
	\includegraphics[width=0.7\linewidth]{Bilder/virtual.png}
	\captionof{figure}[virtual-reality]{Skizze des User Interfaces einer VR Applikation\newline
	Rot: Virtuelle Objekte, Schwarz: Materielle Obkjekte}
	\label{fig:virtual_reality}
\vspace{1em}
\end{minipage}

Bis heute fand die Idee Anklang in Literatur und Film, vom Holodeck in Star Treck bis zur distopischen Zukunft in Matrix. 
In der Realität fand Virtual Reality außerhalb der Forschung vor allem Anwendung in der Unterhaltungsindustrie und im militärischen Kontext. Für Privatpersonen als Videospiele in den Arkaden der 90er Jahre und als Attraktion im Format des 4D-Kinos sowie für das Training von Piloten und Fahrern im militärischen, industriellen Kontext.
\subsubsection{Umsetzung und Stand der Technik}
Für die Realisierung der visuellen Stimulation ist die verbreitetste Option das Tragen einer nach außen geschlossenen Brille mit integrierten Bildschirmen, auf denen die virtuelle Welt abgespielt wird. Problematisch ist hierbei, dass das Auge versucht sich auf Objekte zu fixieren, die entfernt erscheinen, wobei tatsächlich alle Objekte gleich weit entfernt sind, was zu unscharfem sehen führen kann. Des weiteren kann eine Dissonanz zwischen gesehener und durch den Vestibularapparat wahrgenommener Bewegung zu Übelkeit führen, was jedoch durch akurates Headtracking minimiert werden kann. Für haptische und taktile Wahrnehmung sind spezielle Handschuhe vorgesehen, die einerseits die Finger- und Handbewegungen messen und andererseits auch Feedback wie Druck oder Vibration geben können.%TODO Fixen! und die Sonderzeichen rausescapen.
Für die auditive Wahrnehmung ist für die Simulation von räumlichem hören bei bineuralen Aufnahmen ein Stereokopfhörer ausreichend. Allerdings sind auch andere System wie Dolby Surround, DTS oder ähnliches denkbar. Die etablierteste Lösung ist momentan das Oculus Rift Headset. Dieses ist mit zwei Bildschirmen ausgestattet, die jeweils mit 1080x1200 Pixel auflösen und ein diagonales Sichtfeld von 110 Grad bieten, womit die Ränder des Bildschirms nichtmehr sichtbar sind. Des Weiteren sind Kopfhörer integriert, die ein dreidimensionales Hören ermöglichen. Für die Bewegung im virtuellen Raum sind ‚Touch‘ genannte Controller mit Beschleunigungssensoren und Knöpfen und Sticks erhältlich. Hier sind jedoch andere Hersteller, wie z.B. HTC durch die Verwendung von kapazitiven Touch-Sensoren etwas weiter und können auch Fingerbewegungen messen. Auch heutzutage steht bei privaten Anwendungen vor allem die Unterhaltungselektronik im Fokus, aber erste professionelle Anwendungen wie z.B. ‚Tilt Brush‘ von Google, das Zeichnen und das erzeugen plastischer Formen im dreidimensionalen Raum erlaubt, nutzen die neuen Interaktionsmöglichkeiten mit VR. Weitere Vertreter sind das bereits erwähnte HTC Vive, Sony’s Playstation VR und diverse Smartphone basierende Head-Mounted-Displays sowie Forschung und Ausbildungsorientierte Simulatoren, deren Komplexität und hoher Preis sie aber für gewöhnliche Produktivanwendungen unattraktiv macht. Für den Versuch ist auf Grund der weiten Verbreitung und Guten Entwicklungsumgebung der Oculus Rift der Vorzug zu gewähren.
% ----------------------------------------------------------------------------------------------------------
% Augmented Reality
% ----------------------------------------------------------------------------------------------------------
\subsection{Augmented Reality}
\subsubsection{Allgemein}
Unter dem Begriff der “Augmented Reality” (\ac{AR}) (engl.'augmented"~ erweitert, übermäßig'; auch deutsch: erweiterte Realität) wird eine Erweiterung der tatsächlichen, selbstwahrgenommenen Realität durch Zuhilfenahme von virtuellen Elementen durch Technik verstanden. Nach Azuma wird dabei “Realität mit der Virtualität kombiniert”. Die Abgrenzung zur virtuellen Realität oder der Nachbildung eines zuvor aufgezeichneten Videos ist die Echtzeiteinblendung und Verarbeitung der Informationen. Dabei ist die AR nach Milgram und Kishino mehr in den Bereich der Realität einzuordnen, dass durch virtuelle Elemente angereichert und eingebettet wird \cite{Tonnis:2010aa}. Die Virtualität ist nicht nur auf visuelle sondern auf alle Sinneswahrnehmungen möglich, verschränkt sich jedoch meist auf das Visuelle. Weiterhin wird von der AR eine Echtzeitdarstellung bzw. Interaktion erwartet. 
%TODO Quellen einfügen

	Eine beispielhafte Vorstellung der AR wird in Abbildung \ref{fig:augmented_reality} dargestellt. So können dem Anwender zusätzliche visuelle Informationen zu seinem bisherigen Sichtfeld eingeblendet werden. Diese können dabei an Gegenstände angepasst und entsprechend verzerrt sein. Das heißt sie stehen in Bezug zu den realen Objekten und sind daran angepasst. 
	Ein weiterer Vorteil durch AR ergibt sich indem auch Informationen, die nicht wahrnehmbar sind, nun darstellbar sind. Zum Beispiel Infrarotstrahlen oder die zeitliche Verschiebung sog. Differenzbilder.
	

\begin{minipage}{\linewidth}
\vspace{1em}
	\centering
	\includegraphics[width=0.7\linewidth]{Bilder/augmented.png}
	\captionof{figure}[augmented-reality]{Skizze des User Interfaces einer AR Applikation\newline
	Rot: Virtuelle Objekte, Schwarz: Materielle Obkjekte}
	\label{fig:augmented_reality}
\vspace{1em}
\end{minipage}

\subsubsection{Umsetzung und Stand der Technik}
Die Vorstellungen der AR und VR aus Film und Fernsehen beeinflussen auch die nachgeahmte Darstellung der Möglichkeiten in der realen Technik. Vorläufer wie 'Terminator'(1984) oder 'Minority Report'(2002) haben die Anwendungsmöglichkeiten der AR gezeigt und es wird versucht diese so umzusetzen. Dabei sind die Anwendungsfälle vielfältig ausgeprägt.\newline
Die benötigten Komponenten in der AR unterteilen sich in Hardware und Software, bzw. alternativ in Darstellung, Tracking und Eingabe. Die Softwarekomponente wird über 3D-Rendering-Bibliotheken\footcites{bspw. OpenGL und DirectX} realisiert. Die Hardwarekomponente ist sehr unterschiedlich ausgeprägt.
AR wird oft über Brillen realisiert. So bekommen beispielsweise Flugzeugpiloten für Militärjets Informationen über Ziel und ihren Flugzustand und weiter Details über ihren Helm eingeblendet und können so bei verminderter Ablenkung interagieren. Dabei wird Zielerfassung und Koordination über AR und einem HUD im Helm realisiert.\cite{Jenkins:2007aa}. In der aktuellen "Lockheed Martin F-35 Lightning II" wird dem Piloten ein sehr immersives Bild vermittelt. Sechs Kameras an Board des Flugzeuges zeichnen alle Blickwinkel des Flugzeuges auf. Diese werden in Echtzeit verarbeitet und dem Piloten im Helm dargestellt. So kann sich der Pilot im Cockpit umschauen und sozusagen "durch das Flugzeug" hindurch schauen \cite{MOYNIHAN:aa}.
Das Google Glass Projekt ist eines der bekanntesten Consumerprodukte. Dabei werden zusätzliche Informationen über ein Prisma in einem Brillengestell bereitgestellt. Das halbtransparente Display blendet zusätzliche Informationen ein. Über eine eingebaute Kamera können Objekte erkannt und passende Informationen eingeblendet werden. Informationen beziehen sich auf Termine, Kalender, Wetter und andere Events die auch im Smartphone des Nutzers angezeigt werden können. Hier erfolgt eine Erweiterung der Realität, jedoch eine geringe bis keine Immersion im Sinne der Verknüpfung zwischen realen und virtuellen Objekten. 
Eine andere Möglichkeit zur Integration von AR stellt die Hardwareebene Head-Up-Display (\ac{HUD}) dar. So bieten verschiedene Autohersteller für ihre Fahrzeuge ein HUD an um dem Fahrer mehr Informationen zu Straßenzustand, Navigation oder Gefahren zu eröffnen. Ein Modullieferant dafür ist z.B. Daqri. HUD's bestehen im aus einem Optikmodul und einer Projektionsfläche. Dabei ist die Projektionsfläche meist eine halbtransparente Fläche, die sich im Sichtfeld des Anwenders befindet. Die zusätzliche virtuelle Information wird auf diese Fläche projiziert. 
Da HUD, AR-Brillen oder andere Techniken sehr speziell im Anwendungsfall sind und sehr teuer sein kann ist eine Darstellung der AR auch über ein Display möglich. Eine preisgünstigste Realisierung der AR Technik erfolgt mittels Display. Vorhandene Displays zum Beispiel in Smartphones reichen dafür aus. Durch weitere vorhandene Sensoren wie Gyroskop, GPS und Front und Rückkamera kann die genaue Lage des Telefons und das Verhältnis zur Person bestimmt werden.\newline
Um die Immersion für den Anwender möglichst authentisch zu gestalten, erfolgt ein Tracking der Objekte. Die Idee ist die virtuellen Elemente angepasst an die realen Gegenstände zu visualisieren. Dafür muss die Lage des Objekts bekannt sein. Das Tracking erfolgt wahlweise über Stereokameras, Bild- und Mustererkennung, Markertracking oder Inertialtracking \cite{Tonnis:2010aa}.
\subsection{Bewertung}
% TODO: Was soll hier bewertet werden? 
% Ich würde einfach nur ein Fazit am Ende schreiben.

% ----------------------------------------------------------------------------------------------------------
% Virtual Augmented Reality
% ----------------------------------------------------------------------------------------------------------
\subsection{Virtual Augmented Reality}
Virtual Augmented Reality (\ac{VAR}) ist ein Begriff der die Mixed-Reality in Verschiebung zur VR begrenzen soll. Die Grundidee ist folgende: Es soll eine rein virtuelle Welt abgebildet werden, die Objekte und Elemente aus der realen Welt nachbildet und mit Texturen versieht. Zusätzlich zum virtuellen Modell der Realität soll die Möglichkeit der Veränderung und der Manipulatur bestehen. \newline
Nach dem Kontinuum von Milgram und Kishino befindet sich die VAR dabei im Bereich der VR mit Bezügen zur Mixed-Reality. Die Immersion soll maximal werden. Es erfolgt eine Abbildung der Realität mit deren Sinneswahrnehmungen in einer virtuellen Welt. Die Darstellung erfolgt unter anderem über Datenbrillen wie bei der VR. Durch reale Objektbezüge gibt es für den Anwender auch ein Feedback, denn es erfolgt eine haptische Rückmeldung bei der Berührung der Gegenstände. \newline
Ein spezielles Anwendungsszenario ist die Chirurgie in der Medizin. In \cite{OLMOS:2014aa} wird ein Konzept vorgestellt, dass neue Möglichkeiten bei Operationen anbietet. Durch minimalinvasive Eingriffe bei Operationen oder immer kleinere und feinere Operationsverfahren werden die Möglichkeiten der Anschauung und der Orientierung für das medizinische Personal immer schwieriger, da beispielsweise unter der Haut mir einer Kamera und dem Operationsbesteck gearbeitet wird. Zum einen ist die Orientierung schwierig, zum anderen sind die gelieferten Daten z.B. die einer Infrarotkamera nicht hinreichend. Weiterhin nimmt die Komplexität und der Stress bei den Ausführenden zu. Um diese Probleme zu verbessern soll ein virtuelles Modell des Patienten angefertigt werden, dieses wird am Patienten getracked.Bei der Operation trägt das behandelnde Medizinische Team Datenbrillen oder sieht über einen Bildschirm eine Visualisierung wo sie sich mit dem Behandlungswerkzeug im Körper befinden und was zu behandeln ist. \newline
Die VAR bietet somit eine Möglichkeit um Ansichten verständlich zu machen, die bisher nicht anschaubar, nicht leicht verständlich oder nur theoretisch nachvollziehbar waren.
%TODO Bild einfügen
\begin{minipage}{\linewidth}
\vspace{1em}
	\centering
	\includegraphics[width=0.7\linewidth]{Bilder/virtual_augmented.png}
	\captionof{figure}[virtual-reality]{Skizze des User Interfaces einer Virtual-Augmented Reality Applikation\newline
	Rot: Virtuelle Objekte, Schwarz: Materielle Obkjekte, Grün: Virtuelle Abbildungen materieller Objekte}
	\label{fig:virtual_augmented_reality}
\vspace{1em}

\end{minipage}
% ----------------------------------------------------------------------------------------------------------
% Hypothese und formale Beschreibung
% ----------------------------------------------------------------------------------------------------------
\section{Hypothese und formale Beschreibung}
Ziel des Versuchs soll der Vergleich zwischen einer rein virtuellen, von der physichen Welt unabhängigen und einer mit der phyischen Umwelt kompatibel gestalteten virtuellen Arbeitsumgebung sein.
Die inkompatibel gestaltete Umgebung wird in Virtual Reality gestaltet. Für die kompatibel gestaltete Umgebung kämen sowohl Augmented Reality, als auch Virtual Augmented Reality in Frage. Da aber die Möglichkeiten der Virtual Augmented Reality erforscht werden sollen, fällt die Wahl auf diese.
Es gilt herauszufinden, ob eine mit der physischen Umwelt kompatibel Gestaltete Umgebung eine Erleichterung und Verbesserung der Arbeit mit Digitalen Medien bewirkt. Zu diesem Zweck sollen die Probanden einfache Aufgaben an einem virtuellen Arbeitsplatz ausführen. Denkbar wäre z.B. das Abschreiben eines Textes oder das übertragen von Tabellen. Dies würde eine bidirektionale Kommunikation zwischen Mensch und Maschine gewähleisten. Die Qualität und Effizienz, sowie die subjektive Einschätzung der Probanden soll dann ausgewertet werden.
Es bietet sich die Formulierung einer Unterscheidungshypothese an.\\
\vspace{1em}
\begin{center}
\textbf{Hypothese:}
Eine mit der physichen Umgebung kompatibel gestaltete virtuelle Arbeitsumgebung verbessert die Arbeit an digitalen Dokumenten gegenüber einer rein virtuelle Umgebung.
\end{center}
\vspace{1em}
Als Nullhypothese ergibt sich: 'Die Kompatibilät hat bei der Gestaltung einer virtuellen Arbeitsumgebung hat keinen Einfluss auf die Arbeit an digitalen Dokumenten.'\\
Zur Bewertung der Aufgabenbewältigung sollen die Fehleranzahl und die Bearbeitungsgeschwindigkeit als quantitative Kriterien herangezogen werden. Außerdem soll durch Befragung der subjektive Eindruck der Testpersonen in die Bewertung einfließen.
Als abhängige Variablen ergeben sich damit die Fehlerzahll, die Bearbeitungszeit sowie das Ergebnis der Befragung.
Die Abhängige variable ist die Kompatibilität des Systems, d.h. eine binäre variable mit  den Optionen physische Objekte abzubilden oder nicht. Es ergibt sich also ein Einfaktorielles Experiment dessen unabhängige Variable zwei Stufen hat.
% ----------------------------------------------------------------------------------------------------------
% Experiment
% ----------------------------------------------------------------------------------------------------------
\section{Experiment}
\subsection{Design}
Für das Experiment wird eine Gruppe von Probanden in einer virtullen Arbeitsumgebung Daten von einer Quelldatei in eine Zieldatei übertragen. Dabei werden die Quelldaten als Fließtext in einem Fenster ausgegeben und sollen dann per Eingabe auf einer virtuellen Tastatur in der Zieldatei in eine Tabelle geschrieben werden (z.B. Wikipedia-Artikel über ein Land, Tabelle mit Feldern für Bevölkerung Bruttoinlandsprodukt, Fläche, etc.)
Der eigentliche Gedanke, bei der VAR die Umgebung life in die virtuelle Welt zu projezieren, muss aus Kostengründen zunächst hinten angestellt werden. Es wird im Vorfeld des Versuchs ein Modell des Versuchsraums erstellt. Die Probanden sitzen an einem festen Arbeitsplatz, die Kopfbewegungen werden getrackt und so das Modell an der optischen Achse des Probanden ausgerichtet.
Um die Eingabe zu Realisieren kann auf einen Leapmotion Kontroller gesetzt werden, für den es bereits Implementierungen von VR-Keyboards gibt \cite{leap-motion}. Zusammen mit einem Oculus Rift Headset
% ----------------------------------------------------------------------------------------------------------
% Literatur
% ----------------------------------------------------------------------------------------------------------
\renewcommand\refname{Quellenverzeichnis}
\bibliographystyle{myalpha}
\bibliography{bibo}
\pagebreak
% ----------------------------------------------------------------------------------------------------------
% Beispiel-Elemente
% ----------------------------------------------------------------------------------------------------------
\section{Beispiele}
\subsection{Bilder}
Zum Einfügen eines Bildes, siehe Abbildung \ref{fig:osgi}, wird die \textit{minipage}-Umgebung genutzt, da die Bilder so gut positioniert werden können.

\vspace{1em}
\begin{minipage}{\linewidth}
	\centering
	\includegraphics[width=0.7\linewidth]{Bilder/layering-osgi.png}
	\captionof{figure}[OSGi Architektur]{OSGi Architektur\footnotemark }
	\label{fig:osgi}
\end{minipage}
\footnotetext{Quelle: \url{http://www.osgi.org/Technology/WhatIsOSGi}}

\subsection{Tabellen}
In diesem Abschnitt wird eine Tabelle (siehe Tabelle \ref{tab:beispiel}) dargestellt.

\vspace{1em}
\begin{table}[!h]
	\centering
	\begin{tabular}{|l|l|l|}
		\hline
		\textbf{Name} & \textbf{Name} & \textbf{Name}\\
		\hline
		1 & 2 & 3\\
		\hline
		4 & 5 & 6\\
		\hline
		7 & 8 & 9\\
		\hline
	\end{tabular}
	\caption{Beispieltabelle}
	\label{tab:beispiel}
\end{table}

\pagebreak
\subsection{Auflistung}
Für Auflistungen wird die \textit{compactitem}-Umgebung genutzt, wodurch der Zeilenabstand zwischen den Punkten verringert wird.

\begin{compactitem}
	\item Nur
	\item ein
	\item Beispiel.
\end{compactitem}

\subsection{Listings}
Zuletzt ein Beispiel für ein Listing, in dem Quellcode eingebunden werden kann, siehe Listing \ref{lst:arduino}.

\vspace{1em}
\begin{lstlisting}[caption=Arduino Beispielprogramm, label=lst:arduino]
int ledPin = 13;
void setup() {
    pinMode(ledPin, OUTPUT);
}
void loop() {
    digitalWrite(ledPin, HIGH);
    delay(500);
    digitalWrite(ledPin, LOW);
    delay(500);
}
\end{lstlisting}

\subsection{Tipps}
Die Quellen befinden sich in der Datei \textit{bibo.bib}. Ein Buch- und eine Online-Quelle sind beispielhaft eingefügt. [Vgl. %\cite{buch}, \cite{online}]

Abkürzungen lassen sich natürlich auch nutzen (). Weiter oben im Latex-Code findet sich das Verzeichnis.
\pagebreak


\end{document}
